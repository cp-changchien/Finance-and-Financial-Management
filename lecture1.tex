%% Lecture 1 19/01/2021
"qualitative in changing input on output, how the price would change." \par 
\begin{multicols}{2}
\subsection{Economics}
\boxed{\textbf{what is economics?}}
\begin{itemize}
    \item study of the production, distribution, and consumption of goods and services
    \item or can be interpreted as there's a lot of stuffs, how do you get stuffs to people
    \item focused on analysing behaviour and maximising welfare. (positive vs normative)
    \begin{itemize}
        \item positive analysis (factual): how do things work; normative: how should things work.
    \end{itemize}
    \item standard reasons for money:
    \begin{itemize}
        \item store of value: moving value through time. 
        \item medium of exchange: common goods for exchange
        \item unit of account: compare goods, one common denominator for comparison. 
    \end{itemize}
\end{itemize}

\boxed{\textbf{why economics paid attention with finance?}}\par
The financial problem blew up which also affects severely on the real economic, proving there're not unrelated but somehow intertwine, leading to \textit{macro-finance}

\subsection{Finance}
\boxed{\textbf{what is finance?}}
\begin{itemize}
    \item Finance is the study of investments. 
    \item The mean-variance approach (micro-level)
    \item The CAPM (macro-level)
    \item Two basic functions:
    \begin{itemize}
        \item Valuation: objective-independent, how are assets valued? How should they be valued?
    \end{itemize}
\end{itemize}

\subsubsection{Primary Market}
\textbf{primary market} is the financial market where entities such as companies, governments and other institutions obtain funds through the sale of debt and equity-based securities. selling part of the company and in return to share their profit.

\subsubsection{Secondary Market}

\textbf{secondary markets} is the financial market where investors buy and sell securities from other investors (stock exchange)

\subsubsection{Purpose of Financial Markets}

% first reason
\boxed{\textbf{why do we need financial market?}}\par
\begin{enumerate}
    \item Allows trading to offset and reduce risks (e.g. corn production with other investment, income is no longer unpredictable, as corn price might drop, profits from investing can save some downsides)
    \item (Setting Prices) The average aggregated information is reflected in the prices (more people want to sell, price goes down), which is valuable for people outside the market, companies, and policy makers. 
    \item (Transferring Risks) People expose to risk that doesn't want to be can have a place to sell it
\end{enumerate}
It essential for both group of these people (people with info vs people don't like risk) to make a financial market to function. 

\subsubsection{Important Assumptions}
\begin{enumerate}
    \item Agents are selfish
    \item Investors prefer more to less
    \item Investors don't like risk
    investors prefer money now to later
    \item No such thing as a free lunch, shouldn't allowed to create when nothing to start with 
    \item Financial market price to set supply = demand
    \item Risk sharing and frictions are central to financial innovation
    \item Don't say that a model is unrealistic. 
\end{enumerate}

\end{multicols}
