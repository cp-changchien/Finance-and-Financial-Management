Arbitrage is a trading strategy that \textbf{(1)} require NO initial investment. \textbf{(2)} Has \underline{NO negative cash flows} at any time. \textbf{(3)} Has a positive cash flow at at least one time. Arbitrage also implies and solves mispricing. 

\begin{multicols}{2}
Important insight in theoretical finance: (1) True arbitrage cannot exist. (2) Prices would adjust to eliminate it. (3) The no-arbitrage condition.

\subsection{Bid-Ask Spread}
\underline{\textbf{Bid:}} the price at which market maker buys an asset.\\[0.2cm]
\underline{\textbf{Ask:}} the price at which market maker sells an asset.\\[0.2cm]
\underline{\textbf{Mid:}} (Bid+Ask)/2, a way to value holdings.\par 
\underline{The Bid-Ask spread} is normally set base on liquidity and other reasons, for a market maker, they care about (1) Their risk aversion (2) Internal guidelines on inventory (3) Reputations. (4) \textbf{Adverse Selection}. if an investor wished to purchase stocks from a fund and has inside or more research on that specific asset, them the fund might widen their spread to minimise their risk in order to offset this lack of information.

\subsection{Implication of No Arbitrage}
\subsubsection{The Law of One Price (LOOP)}
If two securities have the same payoffs, they have the same price. e.g.\par 

Chase is offering a bond that pays \$100 in one year, at \$94.34.
Merrill Lynch is offering a bond that pays \$100 in one year, at \$95.23. In this arbitrage example, arbitrageurs tends to borrow/sell Merrill Lynch bond and buy the Chase bond to make profit.However transaction costs make it more difficult to find arbitrages, there isn't necessarily a deterministic price with transaction costs.

\subsubsection{Replicating Portfolio}
If a portfolio has the same payoffs as a security, it must have the same price. e.g.\par

If a bond three-year bond with a 10\% coupon rate is trading for \$100. At the same time 
\begin{itemize}
    \item A zero-coupon bond maturing in 1 year costs \$98
    \item A zero-coupon bond maturing in 2 year costs \$96
    \item A zero-coupon bond maturing in 3 year costs \$93
\end{itemize}

From the replicating portfolio, we can refer that \$10 in \underline{one year} is worth \$9.8 today; \$10 in \underline{two year} is worth \$9.6 today, \$110 in \underline{three year} is worth \$102.3 today, the bond should worth \textcolor{red}{\$121.7}, therefore, buy the bond and short the replicating portfolio if it's priced at \$100.

\subsubsection{Dynamic Hedging Strategy}
If a self-financing strategy has the same payoffs as a security, it must have the same price. e.g.\par 

How much should a zero-coupon bond maturing in two years cost if:
\begin{itemize}
    \item A zero-coupon bond maturing in one year costs \$98.
    \item In one year, a one year zero-coupon bond also costs \$98.
\end{itemize}
Need \$100 in two year, so i need \$98 in one year (for the second point), therefore, to get only \$98 in one year, the bond today will cost 0.98*98= \$96.04. If priced at \$95: buy the two-year zero. Short 0.98 units of the 1-year zero at \$96.04. Make \$1.04.

\end{multicols}